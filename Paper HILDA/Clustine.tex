\documentclass{sig-alternate}
\usepackage{eurosym}
\usepackage{framed}

\usepackage{graphicx}
\usepackage{caption}
\usepackage{subcaption}

\usepackage{dirtytalk}


\newenvironment{itemize0}
{ 
    \begin{itemize}
        \setlength{\topsep}{0pt}
        \setlength{\itemsep}{0pt}
        \setlength{\parskip}{0pt}
        \setlength{\parsep}{0pt} 
}
{ \end{itemize}  }

\newtheorem{defi}{Definition}

\begin{document}

\title{Have a Chat with Clustine,\\Conversational Engine to Query Large Tables}

\numberofauthors{2} %  in this sample file, there are a *total*
\author{
\alignauthor
Thibault Sellam\\
       \affaddr{CWI, the Netherlands}\\
       \email{thibault.sellam@cwi.nl}
% 2nd. author
\alignauthor
Martin Kersten\\
       \affaddr{CWI, the Netherlands}\\
       \email{martin.kersten@cwi.nl}
}

\maketitle
\begin{abstract} 
Given the recent advances of AI and the stellar popularity of messaging apps
(e.g., WhatsApp), conversational engines are no longer seen as quirky artifacts
of customer support services and computer science museums. Chatbots provide a
mighty, lightweight and accessible way to propose services over the Internet.
In this paper, we introduce Clustine, a system designed to help users query
large tables through short messages. The main idea is to combine cluster
analysis and text generation to compress the query results, describe then with
natural language and make recommendations. We detail the architecture of our
system, demonstrate it with two use cases, and present early experiments with
10 real datasets to show that its promises are reachable.
\end{abstract}

\section{Introduction}
\label{sec:intro}

For many database users, writing database queries is a struggle. It is a
struggle because mastering a query language requires training. It is also a
struggle because it requires a precise and exhaustive knowledge of the
database. Users must know exactly which table to use, which columns to inspect
and which conditions to set. When users \emph{explore} their data, that is,
when they are interrogating it to discover its content, they do not have this
knowledge.

To address this problem, software editors and researchers have come up
\emph{natural language interfaces} to databases. The idea is to let users
interrogate their databases with plain English. It is then up to the system to
interpret the query and cast it in a form that the underlying data manager can
understand. This approach constitutes a huge leap forward, but it is not exempt
of drawbacks. For a start, it still assumes that they user has a specific query
in mind. Even if they do not know how to express this query SQL, they must have
some idea of which column to use. Furthermore, it only solves half  of the
problem: the users express their queries in SQL, but they still obtain their
results in tabular format. If the output contains a few tuples and a handful of
columns, this paradigm is ideal. But what if the users are interested in
broader queries?

A popular alternative is to use visual tools, such as Tableau. With these
packages, users can write queries with drag and drops and obtain their results
with visualizations. But this approach has its limits too. It does not solve
the starting point problem, as users still need to write a query. Furthermore,
visualizations are no panacea. Their main drawback come from the fact that they
attempt to show everything. Consequently, they require training (though mild),
attention, and they cannot scale beyond a few dozen variables.

In this paper, we introduce Clustine, the first conversational agent for data
exploration. Our system remains on two pillars. First, it uses natural language
during the whole exploration process. This means that it collects queries, but
also \emph{answers} in natural language. While several papers have studied the
first direction, little to none of them have tackled the reverse direction.
Second, our system is proactive: it makes suggestions, instead of collecting
queries passively. It is then up to the user to accept or reject these
suggestions.

This paper is an early-stage report: we present the main ideas behind Clustine
and present preliminary experiments to show that they are feasible.
Nevertheless, we will omit a few details and leave questions open for future
publications. This work is partly inspired by Blaeu, a system to explore data
with cluster analysis. The main difference is that Blaeu relies heavily on
visualizations and expert judgment.

\section{Overview}
\label{sec:overview}

Clustine has several advantages compared to visual tools such as Tableau. It is
proactive, in the sense that it suggests questions to the user. Also, its scope
is ever broader that that of Tableau: it can be used by users with no literacy
in data analysis, user who rely on audio (e.g., visually impaired users), and
it is generally well suited to educational contexts. But the strengths of our
method are also its weaknesses. As Clustine makes ``editorial choices'', it
omits potentially interesting information about the data. Furthermore, full
text is much less space efficient than charts. Therefore, Clustine is
conceptually, but also materially limited.

\section{Architecture}
\label{sec:architecture}

\section{Experiments}
\label{sec:experiments}

\subsection{Accuracy of the Descriptions}
\label{sec:description}

\subsection{Speed of the System}
\label{sec:speed}

\subsection{Simulation}
\label{sec:simulation}

\section{Related Work}
\label{sec:related}

\section{Conclusion}
\label{sec:conclusion}
During the last few years, several companies have introduced virtual
assistants: Cortana, Siri, Google Voice. In this paper, we investigated how to
develop a similar system for data exploration.


\small
\bibliographystyle{abbrv}
\bibliography{Clustine}
\end{document}
