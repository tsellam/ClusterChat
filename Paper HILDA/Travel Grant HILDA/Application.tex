\documentclass{article}
\usepackage{eurosym}
\usepackage{framed}

\usepackage{graphicx}
\usepackage{caption}
\usepackage{subcaption}

\begin{document}

\title{Application for the Paxata Travel Grant}
\date{}

\maketitle

\begin{tabular}{ll}
    Student & Thibault Sellam\\
            & thibault.sellam@cwi.nl\\
            & \\
    Advisor & Prof. Martin Kersten\\
            & martin.kersten@cwi.nl\\
            &\\
    Affiliation & CWI (Dutch Center for Mathematics and Computer Science)\\
                & 123 Science Park\\
                & 1098XG Amsterdam\\
                & The Netherlands\\
                & \\
\end{tabular}
~\\
I a PhD student at CWI, in the Database Architectures group. I started in
October 2011 and I expect to finish in October 2016.\\
~\\
The following submission was accepted for a long talk:\\
\emph{Have a Chat with Clustine,
Conversational Engine to Query Large Tables}\\
Thibault Sellam and Martin Kersten
\pagebreak

\section*{Automatic Assistants for Data Exploration}
In less than five years, so-called data scientists have moved from obscurity to
fame. But those ``unicorns'' are still a rare and expensive bunch. One reason
for their scarcity is  the high entry bar to learning data science. Aspiring
practitioners must learn programming, database tuning, statistical modeling and
data visualization. The training is difficult, it takes years and requires
passion.  I believe that analyzing large volumes of data should be simpler. The
broad aim of my PhD research is to make it available to casual users - that
is, users with neither the time, the training nor the budget to train deep
neural networks on massive GPU clusters.

Concretely, my mission is to design simpler software tools to support data
exploration. To do so, I have developed \emph{automatic assistants}, that
integrate machine learning, visualization and database queries into
ready-to-use, unified interactive systems.  In previous publications, I
presented Blaeu, a system to guide data explorers with subspace search and
cluster analysis~\cite{sellamTKDE, sellam2013meet}.  The main idea was to
create \emph{data maps}, i.e., visual representations of the data in which the
user could zoom or apply projections.  I also developed Claude, a framework to
recommend OLAP queries through feature selection~\cite{Sellam2015Semi}.  The
main challenge was to develop an information theoretic model to help users find
``good'' database views.  In my HILDA submission, I present an early stage
conversational engine to guide users through their tuples. I believe that this
line of research can lead to exciting developments, though much research
remains to be carried out.

The HILDA workshop is the ideal venue to present my work because it targets
exactly my research interests. Attending it will allow me to share ideas and
collect feedback. It will allow me to discover the latest development in data
exploration research. It is also an incredible opportunity for me to meet
world-class experts on the questions that have kept busy for the last five
years.

In conclusion, there exists much better tools than queries and tables to explore
data. The information visualization community has understood this fact decades
ago and has made tremendous progress ever since. But I am convinced that the 
problem is all but closed. The database community can, and should join this
quest. For this reason, I strongly support HILDA's vision. I am 
proud and excited to contribute to this workshop.

\small
\bibliographystyle{abbrv}
\bibliography{Application}
\end{document}
